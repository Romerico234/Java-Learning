\documentclass[12pt,letterpaper,final]{report}
\usepackage[utf8]{inputenc}
\usepackage{amsmath}
\usepackage{amsfonts}
\usepackage{amssymb}
\usepackage{amsthm}
\renewcommand\qedsymbol{$\blacksquare$}
\usepackage{enumerate}
\usepackage{hyperref}
\usepackage{pdfpages}
\usepackage{graphics}
\usepackage{graphicx}
\usepackage{float}
\usepackage{physics}
\usepackage{tikz}
\usepackage{tikz-qtree}
\usetikzlibrary{automata,arrows}
\usepackage{listings}


%\author{Marius Zimand}

\begin{document}


\fbox{
\vbox{
\begin{flushleft}
Romerico David, Haley Elliott, Julian Halsey, Nkechiyem Molokwu\\  % authors' names
COSC 336 \\  %class
02/22/2024\\  % date
\end{flushleft}
\center{\Large{\textbf{Assignment 2}}}
%\end{mdframed}
} % end vbox
} % end fbox
\vline


\noindent\textbf{Exercise 1:}  

\bigskip

\textbf{a.} $\Theta ((4n+1)(4^{log_2(n)}))$

\quad $= \Theta (4n^3+n^2)$

\quad $= \Theta (n^3)$

\bigskip

\textbf{b.} $t_1(n) = n^2 + n$

\quad $t_2(n) = n^2$

\bigskip

\textbf{c.} $t_3(n) = n^2$ since $t_3(2n) = (2n)^2 = 4n^2 = \theta (n^2)$

\bigskip

\textbf{d.}  $t_4(n) = 2^n$ since $t_4(2n) =  2^{2n}$

\bigskip

\noindent\textbf{Exercise 2:} 

\begin{table}[h]
\centering
\begin{tabular}{|c|c|c|c|c|c|c|c|}
\hline
& A & B & $O$ & $o$ & $\Omega$ & $\omega$ & $\Theta$ \\
\hline
a. & $\log^kn$ & $n^\epsilon$ & Yes & Yes & No & No & No  \\
\hline
b. & $n^k$ & $c^n$ & Yes & Yes & No & No & No  \\
\hline
c. & $\sqrt{n}$ & $n^{\sin n}$ & No & No & No & No & No  \\
\hline
d. & $2^n$ & $2^{n/2}$ & No & No & Yes & Yes & No  \\
\hline
e. & $n^{\log c}$ & $c^{\log n}$ & Yes & No & Yes & No & Yes  \\
\hline
f. & $\log(n!)$ & $\log(n^n)$ & Yes & No & Yes & No & Yes \\
\hline
\end{tabular}
\end{table}

\bigskip

\noindent\textbf{Exercise 3:}
\bigskip

\textbf{a.} $(n \cdot n)(\frac{n}{2}) = \frac{n^3}{2}$

\quad $=> \Theta(\frac{n^3}{2})$

\quad $= \Theta(n^3)$

\bigskip

\textbf{b.} $n + \frac{n}{2}$

\quad $=> \Theta(n+\frac{n}{2})$

\quad $= \Theta(n)$

\bigskip

\textbf{c.} $(n \cdot n)(n \cdot n) = n^4$

\quad $=> \Theta(n^4)$

\bigskip

\textbf{d.} $\Theta(\log_2n)$

\bigskip

\textbf{e.} $(n)(log_2(n \cdot n))$

\quad $=> (n)(\log_2(n^2))$

\quad $=> (n)(2\log_2n)$

\quad $= \Theta(n\log_2n)$


\bigskip

\bigskip

\noindent\textbf{Exercise 4:}

\textbf{a.} $S_1 = 500 + 501 + 502 +....+ 999$


\quad $n = 999 - 500 + 1 = 500$, \space $a = 500$, \space $d = 501 - 500 = 1$

\quad 

% \quad $a_n = 999 = a + (n-1)d = 500 + (n-1)1$

% \quad $499 = (n-1)1$

% \quad $499 = n-1$

% \quad $n = 500$

\quad $S_1 = (500)(500) + (\frac{1(500(500-1))}{2})$

\quad $= 250000 + (\frac{249500}{2})$

\quad $= 250000 + 124750$

\quad $= 374750$

\bigskip

\textbf{b.} $S_2 = 1 + 3 + 5 +....+ 999$

\quad $n = \frac{999-1}{2}+1 = 500$,\space$a = 1$,\space$d = 2$\space$

% \quad $a_n = 999 = a + (n-1)d = 1 + (n-1)2$

% \quad $998 = (n-1)2$

% \quad $499 = n-1$

% \quad $n = 500$

\bigskip

\quad $S_2 = (500)(1) = (\frac{2(500(500-1))}{2}$

\quad $= 500 + (500(500 - 1))$

\quad $= 500 + (500(499)$

\quad $= 500 + 249500$

\quad $= 250000$

\bigskip

\textbf{c.} $n = 30$,\space $k = 4$

\quad Number of possible committees $ = \binom{n}{k} = \binom{30}{4}$

\quad $\binom{n}{k} = \frac{n!}{k!(n-k)!}$

\quad  $=> \binom{30}{4} = \frac{30!}{4!(30-4)!} = \frac{30!}{4!26!}$

\quad

\quad $= \frac{30\cdot29\cdot28\cdot\27\cdot26!}{4!26!}$

\quad

\quad $= \frac{30\cdot29\cdot28\cdot27}{24}$

\quad $= 27405$

\quad There are $27405$ possible combinations.

\bigskip

\textbf{d.} $C_n = \binom{n}{4}$

\quad $\binom{n}{4} = \frac{n!}{4!(n-4)!}$

\quad $= \frac{n(n-1)(n-2)(n-3)(n-4)(n-5)...}{4!\cdot(n-4)(n-5)...}$

\quad $= \frac{n(n-1)(n-2)(n-3)}{4!}$

\quad $=\frac{n^4-6n^3+11n^2-6n}{24}$

\quad = $\Theta(n^4)$


% = \frac{1}{4!}\frac{n!}{(n-4)!}$

% \quad $ = \frac{1}{4!}(n*(n-1)*(n-2)*....*(n-5))$ 

% \quad $\frac{1}{4!}$ is a constant, so $\frac{1}{4!} = \Theta(1)$

% \quad $(n*(n-1)*(n-2)*....*(n-5) = n^4(1*(1-\frac{1}{n})*(1-\frac{2}{n})*....*(1-\frac{4}{n}))$

% \quad $4$ is a constant $\therefore$\space as $n$ approaches $\infty$, $(1*(1-\frac{1}{n})*(1-\frac{2}{n})*....*(1-\frac{4}{n}))$

% \quad will approach $1$.

% \quad So, $n^4(1*(1-\frac{1}{n})*(1-\frac{2}{n})*....*(1-\frac{4}{n})) = n^4(1) = n^4$

\quad $C_n = \Theta(n^4)$ \space $\therefore$ \space $C_n \neq o(n^3)$ 

\bigskip


\noindent\textbf{Exercise 5:}

Lower Bound:

$S_n \ge \int_{0}^{n} x^2\sqrt{x} dx$

$= \int_{0}^{n} x^{5/2}$

$=\eval{\frac{2}{7}x^{7/2}}_0^n $

$= \frac{2}{7}n^{7/2}$

$=n^{7/2}$


\bigskip
Upper Bound:

$S_n \le \int_{1}^{n+1} x^2\sqrt{x} dx$

$= \int_{1}^{n+1} x^{5/2}$

$=\eval{\frac{2}{7}x^{7/2}}_1^{n+1}$

$= \frac{2}{7}(n+1)^{7/2}-\frac{2}{7}$

$= \frac{2}{7}(n+1)^{7/2}$

$= (n+1)^{7/2}$

$=n^{7/2}$
\bigskip

$\therefore \Theta(n^{7/2})$
\bigskip

% For union of sets, write $A \cup B$; for intersection $A \cap B$; the empty set is denoted $\emptyset$.




% Greek letters are easy to write: $\alpha$, $\beta$, $\gamma$, $\theta$, $\Theta$, $\omega$, $\Omega$, and so on. 

% See \url{https://artofproblemsolving.com/wiki/index.php/LaTeX:Symbols} for a large list of latex symbols.

% This is how we can write math equations on a separate line:
% \[
% a+b = c^2 + \log n. 
% \]

% This is how we can write math equations on a separate line, which include normal text
% \[
% a+b = c^2 + \log n. \mbox{    (Joe's equation).}
% \]

% This is how we can write multi-line math equations on a separate line:
% \[
% \begin{array}{ll}
% a+b & = c^2 + \log n \\
% &\leq 5d + \sin x \\
% & = A.
% \end{array}
% \]
% Matrices can be written like this:
% \[
% A = 
% \begin{pmatrix}
% 1 & a & b \\
% 2 & a^2 & b^3 \\
% 3 & a_3 & b_5
% \end{pmatrix}
% \]
% \bigskip

% This is how to make a table:

% \medskip

% \begin{tabular}{|c|c|c|}
% \hline
% $\delta$ & $a$ & $x$ \\
% \hline
% $q_{1}$ & $q_{1}$ & $ b$ \\
% $q_{2}$ & $q_{1}$ & $ c$ \\
% $q_{3}$ & $q_{2}$ & $ b$ \\
% $q_{4}$ & $q_{3}$ & $ c$ \\
% $q_{5}$ & $q_{4}$ & $ b$ \\
% \hline
% \end{tabular}
% \medskip

% This is how to draw the diagram of an automaton:
% \medskip

% \begin{tikzpicture}[>=stealth',shorten >=1pt,auto,node distance=2.8cm]]

%   \node[initial, state, accepting] (3) {$q_{3}$};
%   \node[state] (2) [above right of=3] {$q_{2}$};
%   \node[state] (4) [below right of=3] {$q_{4}$};
%   \node[state, accepting] (1) [right of=2] {$q_{1}$};
 
  
%   \path[->]
%     (3.north) edge [left] node {$u$} (2.west)    
%     (2) edge [right] node {$d$} (3)
    
%     (3) edge [right] node{$d$} (4)
%     (4.west) edge [left] node {$u$} (3.south)
    
%     (2) edge [bend left] node[above] {$u$} (1)
%     (1) edge [loop right] node {$u$} (1)
%     (1) edge [bend left] node[below] {$d$} (2)
%        ;  
% \end{tikzpicture}

% This is how to draw an arbitrary graph using a package called TikZ.
% \medskip


% \begin{tikzpicture}
% %% vertices
% \draw[fill=black] (0,0) circle (3pt);
% \draw[fill=black] (4,0) circle (3pt);
% \draw[fill=black] (2,1) circle (3pt);
% \draw[fill=black] (2,3) circle (3pt);
% %% vertex labels
% \node at (-0.5,0) {1};
% \node at (4.5,0) {2};
% \node at (2.5,1.2) {3};
% \node at (2,3.3) {4};
% %%% edges
% \draw[thick] (0,0) -- (4,0) -- (2,1) -- (0,0) -- (2,3) -- (4,0) -- (2,1) -- (2,3);
% \end{tikzpicture}

% This should be pretty self-explanatory. The ordered pairs in parentheses are all simply coordinates in the plane.


% The edges are drawn by taking a walk through the graph, using each edge exactly once.
% You can do much fancier things in TikZ, but this should at least get you started.




% \bigskip

% To make latex produce the text exactly how we type, we can use the verbatim environment. This is useful for example to write an algorithm in pseudo-code. Below is a short example.

% \begin{verbatim}
% s = 0
% for i going from 1 to n

%    s= s+ a[i]

% end-for
% \end{verbatim}

% To append in the  latex file the Java source code and the screenshots, you can print them  as a pdf file (placed in the same folder) and then include them  in the latex file with 
% \begin{verbatim}
% \includepdf[pages=-,pagecommand={},width=\textwidth]{file.pdf}

% \end{verbatim}
% To append in the  latex file a .jpg file (for a photo), use 
% \begin{verbatim}
% \includegraphics[width=\linewidth]{photo.jpg}

% \end{verbatim}




% \noindent\textbf{Problem 2:} Here is the solution for problem 2.....





% \bigskip

% \noindent\textbf{Problem 3:} Here is the solution for problem 3. ...


% \bigskip

% \noindent\textbf{Problem 4:}  Here is the solution for problem 4. ...

\end{document}